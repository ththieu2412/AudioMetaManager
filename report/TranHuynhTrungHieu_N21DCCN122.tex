\documentclass[conference]{IEEEtran}
\usepackage[utf8]{vietnam}
\usepackage{cite}
\usepackage{amsmath,amssymb,amsfonts}
\usepackage{graphicx}
\usepackage{textcomp}

% Change section numbering to Roman numerals with a dot
\renewcommand{\thesection}{\Roman{section}.}

% Ensure Vietnamese hyphenation works properly
\hyphenation{}

\begin{document}

\title{Trích xuất và Quản lý Metadata Âm thanh bằng Python}

\author{
    \IEEEauthorblockN{Trần Huỳnh Trung Hiếu}
    \IEEEauthorblockA{
        \textit{Khoa Công Nghệ Thông Tin} \\
        \textit{Học viện Công nghệ Bưu chính Viễn thông cơ sở tại Thành phố Hồ Chí Minh} \\
        TP. Hồ Chí Minh, Việt Nam \\
        Lớp D21CQCNHT01-N \\
        n21dccn122@student.ptithcm.edu.vn
    }
}

\maketitle

\begin{abstract}
Trong lĩnh vực cơ sở dữ liệu đa phương tiện, việc quản lý và truy xuất metadata của các tệp âm thanh là một yêu cầu quan trọng để tối ưu hóa việc tìm kiếm và phân loại dữ liệu. Đề tài này trình bày một giải pháp sử dụng Python để trích xuất và quản lý metadata âm thanh, nhằm hỗ trợ việc tổ chức và lưu trữ thông tin âm thanh trong các hệ thống cơ sở dữ liệu. Phương pháp được triển khai sử dụng thư viện Mutagen để trích xuất thông tin metadata như tên bài hát, ca sĩ, album, và thời gian. Các dữ liệu này sau đó được lưu trữ trong cơ sở dữ liệu SQLite, cho phép dễ dàng truy vấn và quản lý. Thử nghiệm cho thấy hệ thống có khả năng trích xuất chính xác thông tin từ các tệp âm thanh với tốc độ nhanh, đồng thời dễ dàng mở rộng để hỗ trợ các định dạng âm thanh khác nhau. Giải pháp này giúp tự động hóa quá trình quản lý metadata và nâng cao hiệu quả trong việc xử lý và tìm kiếm dữ liệu âm thanh.
\end{abstract}


\section{GIỚI THIỆU}


\section{MÔ TẢ VẤN ĐỀ}


\section{CÁC GIẢI PHÁP ĐỀ XUẤT}


\section{TIÊU CHÍ ĐÁNH GIÁ CÁC GIẢI PHÁP}


\section{PHƯƠNG PHÁP NGHIÊN CỨU}


\section{PHÂN TÍCH VÀ DIỄN GIẢI DỮ LIỆU THU ĐƯỢC}


\section{KẾT LUẬN VÀ ĐỀ XUẤT}

\section{DỮ LIỆU BỔ SUNG}

\section{TÀI LIỆU THAM KHẢO}

\end{thebibliography}

\end{document}